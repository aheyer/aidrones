\renewcommand{\deutschertitel}{Follow-Me-Funktion}
\renewcommand{\englischertitel}{Follow-Me Function}
\chapter[\texorpdfstring{\protect{\vspace{2pt}\englischertitel}}{\englischertitel}]{}
\kapitel{\deutschertitel}
\thispagestyle{empty}

\label{KapitelFollowme}

\begin{paracol}{2}[]

{\raggedright\huge\bfseries\sffamily \englischertitel \par\ } \\[1.8ex]

\switchcolumn

{\raggedright\huge\bfseries\sffamily \deutschertitel \par\ } \\[1.8ex]

\coleng

The Follow-Me function allows an FPV drone to track and follow a moving target, typically a person carrying a GPS device, a smartphone, or a companion transmitter. This capability is useful for recording moving subjects, supporting inspection tasks, or enabling semi-autonomous flight operations. 

\colger

Die Follow-Me-Funktion ermöglicht es einer FPV-Drohne, einem sich bewegenden Ziel zu folgen, typischerweise einer Person, die ein GPS-Gerät, ein Smartphone oder einen zusätzlichen Sender mit sich führt. Diese Fähigkeit wird unter anderem zur Aufnahme bewegter Szenen, zur Unterstützung von Inspektionsaufgaben oder für teilautonome Fluganwendungen genutzt. 

\coleng

Depending on the firmware, the Follow-Me function may rely on external GPS data, telemetry links, onboard sensors, or a combination of these. The following sections describe how Follow-Me is implemented in Betaflight, iNAV, ArduPilot, and provide an outlook on its implementation in PX4. 

\colger

Je nach Firmware basiert die Follow-Me-Funktion auf externen GPS-Daten, Telemetrieverbindungen, an Bord befindlichen Sensoren oder einer Kombination dieser Quellen. Die folgenden Abschnitte beschreiben die Umsetzung der Follow-Me-Funktion in Betaflight, iNAV, ArduPilot sowie einen Ausblick auf die Implementierung in PX4. 

\colende

\renewcommand{\deutschertitel}{Follow-Me mit Betaflight}
\renewcommand{\englischertitel}{Follow-Me with Betaflight}
\makroabschnitt
\label{AbschnittFollowMeBetaflight}

Betaflight does not provide a Follow-Me mode, as the firmware is primarily optimized for manual flight (freestyle and racing). Only basic GPS-assisted functions such as rescue mode are available. Follow-Me operation is not supported because Betaflight lacks autonomous navigation, positional tracking, and target-following logic. Implementing Follow-Me functionality would require a transition to iNAV or ArduPilot.

\colger

Betaflight bietet keinen Follow-Me-Modus, da die Firmware primär für für manuelles Fliegen (Freestyle und Rennen) ausgelegt ist. Lediglich grundlegende GPS-gestützte Funktionen wie der Rescue Mode sind verfügbar. Ein Follow-Me-Betrieb wird nicht unterstützt, da Betaflight keine autonome Navigation, keine Positionsverfolgung und keine Zielverfolgungslogik bereitstellt. Für den Einsatz der Follow-Me-Funktion wird daher ein Wechsel zu iNAV oder ArduPilot erforderlich.

\colende

\renewcommand{\deutschertitel}{Follow-Me mit iNAV}
\renewcommand{\englischertitel}{Follow-Me with iNAV}
\makroabschnitt
\label{AbschnittFollowMeINAV}

iNAV provides a dedicated Follow-Me mode that uses the GPS position of an external device, typically the radio transmitter equipped with an external GPS module or a compatible telemetry system. 

\colger

iNAV stellt einen eigenen Follow-Me-Modus bereit, der die GPS-Position eines externen Geräts verwendet, üblicherweise die Fernsteuerung mit einem angeschlossenen GPS-Modul oder einem kompatiblen Telemetriesystem. Für Follow-Me in iNAV werden ein Flugcontroller mit GPS, Barometer, magnetischen Kompass sowie eine stabile GPS-Empfangsqualität benötigt.

\coleng

Follow-Me in iNAV requires a flight controller with GPS, barometer, magnetic compass, and stable GPS reception. The external device transmits its position to the flight controller via a telemetry protocol such as MAVLink, MSP, or a proprietary transmitter protocol. 

\colger

Das externe Gerät übermittelt seine Position über ein Telemetrieprotokoll wie MAVLink, MSP oder ein proprietäres Fernsteuerungsprotokoll an den Flugcontroller. 

\coleng

Configuration is performed in the iNAV Configurator by enabling the Follow-Me mode, assigning it to an AUX channel, and configuring the GPS update rate and safe distances. iNAV supports various follow patterns, including direct pursuit and offset tracking.

\colger

Die Konfiguration erfolgt im iNAV Configurator durch Aktivieren des Follow-Me-Modus, das Zuweisen eines AUX-Kanals sowie die Einstellung der GPS-Aktualisierungsrate und der Sicherheitsabstände. iNAV unterstützt unterschiedliche Verfolgungsprofile, darunter direkte Verfolgung und seitlich versetztes Tracking.

\colende

\renewcommand{\deutschertitel}{Follow-Me mit ArduPilot}
\renewcommand{\englischertitel}{Follow-Me with ArduPilot}
\makroabschnitt
\label{AbschnittFollowMeArduPilot}

ArduPilot provides a comprehensive Follow-Me implementation that supports external GPS devices, companion computers (e.g., Raspberry Pi Zero, Raspberry Pi Compute Module\,4/5, NVidia TX1/TX2), and smartphone applications via MAVLink. Follow-Me requires a GPS-equipped external device. The drone must include a barometer, magnetic compass, accelerometer, gyroscope, and GPS module. 

\colger

ArduPilot bietet eine umfassende Follow-Me-Implementierung, die externe GPS-Geräte, Companion-Computer (z.B. Raspberry Pi Zero, Raspberry Pi Compute Module\,4/5, NVidia TX1/TX2) und Smartphone-Anwendungen über MAVLink unterstützt. Für Follow-Me wird ein externes GPS-Gerät benötigt. Die Drohne muss über Barometer, magnetischen Kompass, Beschleunigungssensor, Gyroskop und ein GPS-Modul verfügen. 

\coleng

Follow-Me is enabled in Mission Planner by selecting the Follow-Me feature and configuring parameters such as follow distance, altitude offset, and update frequency. 

\colger

Die Aktivierung erfolgt im Mission Planner über die Follow-Me-Funktion, wobei Parameter wie Verfolgungsabstand, Höhenversatz und Aktualisierungsrate konfiguriert werden. 

\coleng

ArduPilot supports advanced behaviors including smooth pursuit, orbiting around the target, and terrain-following when additional sensors are available. Its implementation is suitable for professional and research applications.

\colger

ArduPilot unterstützt fortgeschrittene Verfolgungsverfahren, darunter sanfte Annäherung, Umkreisung des Ziels und Geländefolgeflug bei Verwendung zusätzlicher Sensoren. Die Umsetzung eignet sich für professionelle und forschungsorientierte Anwendungen.

\colende

\renewcommand{\deutschertitel}{Follow-Me mit PX4}
\renewcommand{\englischertitel}{Follow-Me with PX4}
\makroabschnitt
\label{AbschnittFollowMePX4}

PX4 provides a Follow-Me feature primarily through the software QGroundControl and companion-computer applications. The function relies on the transmission of position data from a smartphone or companion system (e.g., Raspberry Pi Compute Module\,4) using MAVLink. PX4 requires a GPS-equipped flight controller with barometer, and magnetic compass. 

\colger

PX4 stellt eine Follow-Me-Funktion bereit, die hauptsächlich über die Software QGroundControl und Companion-Computer-Anwendungen implementiert wird. Die Funktion basiert auf der Übertragung von Positionsdaten eines Smartphones oder eines Companion-Systems (z.B. Raspberry Pi Compute Module\,4) über MAVLink. Für den Follow-Me-Betrieb benötigt PX4 einen GPS-fähigen Flugcontroller mit Barometer und magnetischen Kompass.

\colende
