\renewcommand{\deutschertitel}{Hardware}
\renewcommand{\englischertitel}{Hardware}

% !!! Das hier war vorher !!!
% \chapter[\englischertitel]{\englischertitel\newline\deutschertitel}
% !!! Das hier war vorher !!!


% Das vspace fügt im Inhaltsverzeichnis einen kleinen Abstand unter dem Kapiteleintrag ein.
% Beim deutschen Inhaltsverzeichnis ist es im book.tex an nur einer Stelle in der Zeile 
% \addcontentsline{deutschestoc}{chapter}{\protect{\vspace{2pt}\thechapter}~#1}}
\chapter[\protect{\vspace{2pt}\englischertitel}]{}
\kapitel{\deutschertitel}

\label{KapitelHardware}

\begin{paracol}{2}[]

{\raggedright\huge\bfseries\sffamily \englischertitel \par\ } \\[1.8ex]

\switchcolumn

{\raggedright\huge\bfseries\sffamily \deutschertitel \par\ } \\[1.8ex]

\switchcolumn*
\selectlanguage{english}

TBD

\switchcolumn
\selectlanguage{ngerman}

TBD

\end{paracol}

\renewcommand{\deutschertitel}{Rahmen}
\renewcommand{\englischertitel}{Frames}

\makroabschnitt

\label{AbschnittFrames}

TBD

\switchcolumn
\selectlanguage{ngerman}

Der Rahmen aus verbindet alle Komponenten der Drohne. Das verwendete Material ist üblicherweise Carbon. Dabei handelt es sich um einen leichtgewichtigen und dennoch hochfesten und verwindungssteifen Verbundwerkstoff aus Kohlenstofffasern. Seltener kommen auch Rahmen aus Kunststoff zum Einsatz. Der Rahmen definiert die Propellergröße (siehe Abschnitt~\ref{AbschnittPropeller}).

\switchcolumn*
\selectlanguage{english}

TBD

\switchcolumn
\selectlanguage{ngerman}

Der Rahmen nimmt üblicherweise zentral die wichtigsten elektronischen Komponenten wie Flugcontroller, Videosender, Empfänger und Kamera auf, um diese zu schützen. Der Akku befindet sich üblicherweise oben auf der Drohne, um Beschädigungen beim Landen zu vermeiden. 

\switchcolumn*
\selectlanguage{english}

TBD

\switchcolumn
\selectlanguage{ngerman}

Wichtige Unterscheidungskriterien bei der Auswahl des passenden Rahmens sind auch die Abstände der Bohrlöcher zur Befestigung des Flugcontrollers und des Videosenders. Gängige Maße sind: 

\switchcolumn*
\selectlanguage{english}

TBD

\switchcolumn
\selectlanguage{ngerman}

\begin{itemize}
\item 30,5 x 30,5\,mm
\item 25,5 x 25,5\,mm
\item 20 x 20\,mm
\end{itemize}


\switchcolumn*
\selectlanguage{english}

TBD

\switchcolumn
\selectlanguage{ngerman}

Verfügt ein Rahmen nicht über passende Bohrlöcher für den ausgewählten Flugcontrollers und den Videosender, kann ein per 3D-Drucker gedruckte Adapter helfen, wenn der Platz im Rahmen dafür ausreicht. 



\end{paracol}






\renewcommand{\deutschertitel}{Flugcontroller}
\renewcommand{\englischertitel}{Flight Controller}

\makroabschnitt

\label{AbschnittFC}

TBD

\switchcolumn
\selectlanguage{ngerman}

TBD

\end{paracol}



\renewcommand{\deutschertitel}{Videosender}
\renewcommand{\englischertitel}{Video Transmitter}

\makroabschnitt

\label{AbschnittVTX}

TBD

\switchcolumn
\selectlanguage{ngerman}

TBD

\end{paracol}




\renewcommand{\deutschertitel}{Propeller}
\renewcommand{\englischertitel}{Propeller}

\makroabschnitt

\label{AbschnittPropeller}

TBD

\switchcolumn
\selectlanguage{ngerman}

TBD

\end{paracol}
