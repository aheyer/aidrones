\renewcommand{\deutschertitel}{KI-Drohnen-Projekte}
\renewcommand{\englischertitel}{AI Drone Projects}

% !!! Das hier war vorher !!!
% \chapter[\englischertitel]{\englischertitel\newline\deutschertitel}
% !!! Das hier war vorher !!!


% Das vspace fügt im Inhaltsverzeichnis einen kleinen Abstand unter dem Kapiteleintrag ein.
% Beim deutschen Inhaltsverzeichnis ist es im book.tex an nur einer Stelle in der Zeile 
% \addcontentsline{deutschestoc}{chapter}{\protect{\vspace{2pt}\thechapter}~#1}}
\chapter[\protect{\vspace{2pt}\englischertitel}]{}
\kapitel{\deutschertitel}

\label{KapitelKI}

\begin{paracol}{2}[]

{\raggedright\huge\bfseries\sffamily \englischertitel \par\ } \\[1.8ex]

\switchcolumn

{\raggedright\huge\bfseries\sffamily \deutschertitel \par\ } \\[1.8ex]

\coleng

TBD

\colger

Dieses Kapitel beschreibt KI-Anwendungen, die FPV-Drohnen zur Datenerfassung und/oder als Tarnsportvehikel nutzen. Vorgestellt werden Anwendungen, die in Forschungsprojekten, Lehrveranstaltungen und Abschlussarbeiten in der Lehreinheit Informatik an der Frankfurt University of Applied Sciences entwickelt, implementiert und evaluiert wurde. Das Kapitel beschreibt zu jeder dieser KI-Anwendungen die benötigten zusätzlichen Hard- und Softwarekomponenten, nötige Schritte zur Realisierung und die Kosten.

\colende

\renewcommand{\deutschertitel}{Objekterkennung}
\renewcommand{\englischertitel}{Object Detection}
\makroabschnitt
\label{AbschnittObjekterkennung}

TBD

\colger

Objekterkennung ist eine der bekanntesten KI-Anwendungen. Zu den bekanntesten quelloffenen Softwarelösungen, die Objekterkennung ermöglichten, gehören das Framework TensorFlow (Lite) für Maschinelles Lernen in Zusammenarbeit mit der Bibliothek OpenCV (\textsl{Open Computer Vision}) zur Bildverarbeitung und Objekterkennung. Eine alternative Lösung ist das Objekterkennungs-Framework (\textsl{You Only Look Once}).

\coleng

TBD

\colger

Prinzipiell ist es möglich die nötige Hard- und Software als zusätzliche Komponenten in der Drohne zu integrieren und diese Komponenten mitfliegen zu lassen. Alternativ kann auch das Livebild am Boden erfasst und zur weiteren Verarbeitung in einen Computer geleitet werden. Eine einfache Möglichkeit, auf das Livebild zuzugreifen, bietet die eventuell vorhandene AV-Schnittstelle der Videobrille. Das Videosignal kann mit Hilfe eines Videograbbers digitalisiert und an einen Computer zur Verarbeitung weitergeleitet werden.

\coleng

TBD

\colger

Abbildung~\ref{AbbildungKompoentenEinerDrohneMitRaspberryPiundCoral} zeigt die Komponenten der FPV-Drohne aus Abbildung~\ref{AbbildungKompoentenEinerDrohneOhneKI}, erweitert um die zur lokalen Bilderkennung nötigen Komponenten, nämlich den Raspberry Pi Einplatinencomputer, ein Kameramodul, den Google Coral TPU Accelerator und einen BEC zur Umwandlung der elektrischen Spannung des Akkus in 5\,V.

\colende

\begin{figure}[htb]
  \centering
    \includegraphics[width=\linewidth]{Komponenten_der_Drohne_Diagramm_v1_en.pdf}
  \caption{Components of a FPV Drone with additional Hardware Components for Object Detection by the Drone itself}
  \label{AbbildungKompoentenEinerDrohneMitRaspberryPiundCoral}
\end{figure}

\renewcommand{\deutschertitel}{Autopilot}
\renewcommand{\englischertitel}{Auto Pilot}
\makroabschnitt
\label{AbschnittAutopilot}

TBD

\colger

TBD

\colende



\renewcommand{\deutschertitel}{Follow-Me-Function}
\renewcommand{\englischertitel}{Follow Me Function}
\makroabschnitt
\label{AbschnittAutopilot}

TBD

\colger

TBD

\colende
