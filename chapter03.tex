\renewcommand{\deutschertitel}{KI-Drohnen-Projekte}
\renewcommand{\englischertitel}{AI Drone Projects}

% !!! Das hier war vorher !!!
% \chapter[\englischertitel]{\englischertitel\newline\deutschertitel}
% !!! Das hier war vorher !!!


% Das vspace fügt im Inhaltsverzeichnis einen kleinen Abstand unter dem Kapiteleintrag ein.
% Beim deutschen Inhaltsverzeichnis ist es im book.tex an nur einer Stelle in der Zeile 
% \addcontentsline{deutschestoc}{chapter}{\protect{\vspace{2pt}\thechapter}~#1}}
\chapter[\protect{\vspace{2pt}\englischertitel}]{}
\kapitel{\deutschertitel}

\label{KapitelKI}

\begin{paracol}{2}[]

{\raggedright\huge\bfseries\sffamily \englischertitel \par\ } \\[1.8ex]

\switchcolumn

{\raggedright\huge\bfseries\sffamily \deutschertitel \par\ } \\[1.8ex]

\coleng

TBD

\colger

Dieses Kapitel beschreibt KI-Anwendungen, die FPV-Drohnen zur Datenerfassung und/oder als Tarnsportvehikel nutzen. Vorgestellt werden Anwendungen, die in Forschungsprojekten, Lehrveranstaltungen und Abschlussarbeiten in der Lehreinheit Informatik an der Frankfurt University of Applied Sciences entwickelt, implementiert und evaluiert wurde. Das Kapitel beschreibt zu jeder dieser KI-Anwendungen die benötigten zusätzlichen Hard- und Softwarekomponenten, nötige Schritte zur Realisierung und die Kosten.

\colende

\renewcommand{\deutschertitel}{Objekterkennung}
\renewcommand{\englischertitel}{Object Detection}
\makroabschnitt
\label{AbschnittObjekterkennung}

TBD

\colger

TBD

\colende


\renewcommand{\deutschertitel}{Autopilot}
\renewcommand{\englischertitel}{Auto Pilot}
\makroabschnitt
\label{AbschnittAutopilot}

TBD

\colger

TBD

\colende



\renewcommand{\deutschertitel}{Follow-Me-Function}
\renewcommand{\englischertitel}{Follow Me Function}
\makroabschnitt
\label{AbschnittAutopilot}

TBD

\colger

TBD

\colende
