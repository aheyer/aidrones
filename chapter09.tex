\renewcommand{\deutschertitel}{Regulatorische und gesetzliche Grundlagen}
\renewcommand{\englischertitel}{Regulatory and Legal Framework}
\chapter[\texorpdfstring{\protect{\vspace{2pt}\englischertitel}}{\englischertitel}]{}
\kapitel{\deutschertitel}
\thispagestyle{empty}

\label{KapitelRegulatorischeGrundlagen}

\begin{paracol}{2}[]

{\raggedright\huge\bfseries\sffamily \englischertitel \par\ } \\[1.8ex]

\switchcolumn

{\raggedright\huge\bfseries\sffamily \deutschertitel \par\ } \\[1.8ex]

\coleng

The operation of drones is not merely a technical issue but is strongly shaped by regulatory and legal frameworks. These regulations define under which conditions drones may be operated, which technical and organizational requirements must be met, and which responsibilities apply to operators and pilots. In the context of research and higher education, these rules are of particular importance, as they directly influence the selection of suitable use cases, the design of teaching formats, and the practical execution of flight experiments. The following section therefore first examines the situation in Germany, which in recent years has been largely shaped by European regulations and supplemented by national provisions.

\colger

Der Betrieb von Drohnen ist nicht nur eine technische Fragestellung, sondern in hohem Maße durch regulatorische und gesetzliche Rahmenbedingungen geprägt. Diese legen fest, unter welchen Voraussetzungen Drohnen eingesetzt werden dürfen, welche technischen und organisatorischen Anforderungen zu erfüllen sind und welche Verantwortlichkeiten Betreiberinnen und Betreiber tragen. Insbesondere im Kontext von Forschung und Lehre sind diese Vorgaben von zentraler Bedeutung, da sie die Auswahl geeigneter Einsatzszenarien, die Gestaltung von Lehrveranstaltungen sowie die praktische Durchführung von Flugversuchen maßgeblich beeinflussen. Im Folgenden wird daher zunächst die Situation in Deutschland betrachtet, die seit einigen Jahren stark durch europäische Vorgaben geprägt ist und durch nationale Regelungen ergänzt wird.

\colende

\renewcommand{\deutschertitel}{Situation in Deutschland}
\renewcommand{\englischertitel}{Situation in Germany}
\makroabschnitt
\label{AbschnittRegulatorischeGrundlagenDeutschland}

The legal framework for operating drones in Germany is clearly structured but complex and cannot be fully covered in all details at this point. Nevertheless, an early and basic understanding of the applicable regulations is essential for their use in research and teaching.

\colger

Der rechtliche Rahmen für den Betrieb von Drohnen in Deutschland ist klar strukturiert, jedoch komplex und kann an dieser Stelle nicht bis ins letzte Detail dargestellt werden. Für den Einsatz in Forschung und Lehre ist eine frühzeitige und grundlegende Auseinandersetzung mit den rechtlichen Rahmenbedingungen unerlässlich.

\coleng

Since 2021, the operation of drones (unmanned aircraft systems -- UAS) in Germany has been largely governed by harmonized European regulations. These are based on EU law and are supplemented by national provisions. The aim of this regulatory framework is to ensure safe operation, minimize risks to people and infrastructure, and at the same time enable innovation. In addition to EU law, national regulations such as the German Air Traffic Act (Luftverkehrsgesetz, LuftVG), the Air Traffic Regulations (Luftverkehrs-Ordnung, LuftVO), and federal state regulations on no-fly zones apply.

\colger

Der Betrieb von Drohnen (unbemannte Luftfahrtsysteme -- UAS) in Deutschland ist seit dem Jahr 2021 weitgehend durch europaweit harmonisierte Regelungen bestimmt. Grundlage ist das EU-Recht, das durch nationale Vorschriften ergänzt wird. Ziel dieser Regulierung ist es, einen sicheren Betrieb zu gewährleisten, Risiken für Personen und Infrastruktur zu minimieren und zugleich Innovation zu ermöglichen. Ergänzend zum EU-Recht gelten nationale Regelungen wie das Luftverkehrsgesetz (LuftVG), die Luftverkehrs-Ordnung (LuftVO) sowie länderspezifische Regelungen zu Flugverbotszonen.

\coleng

Central to the European framework are Implementing Regulations (EU) 2019/947 (rules for UAS operations) and (EU) 2019/945 (technical requirements for drones). These regulations classify drone operations into three categories: Open, Specific, and Certified. In higher education and research, drones are typically operated within the Open Category, which is intended for low-risk operations without the need for individual operational authorization. The Open Category is further divided into the subcategories A1, A2, and A3, which define whether flights over uninvolved persons or assemblies of people are permitted and which minimum distances must be maintained.

\colger

Zentral sind die EU-Durchführungsverordnungen (EU) 2019/947 (Betriebsvorschriften) und (EU) 2019/945 (technische Anforderungen an Drohnen). Diese teilen Drohnenbetriebe in drei Kategorien ein: Open, Specific und Certified. In der Hochschullehre und Forschung findet der Einsatz in der Regel in der sogenannten Offenen Kategorie (Open Category) statt, die für risikoarme Flüge ohne individuelle Genehmigung vorgesehen ist. Die Offene Kategorie definiert die Unterkategorien A1, A2 und A3, die jeweils festlegen, ob einzelne unbeteiligte Personen oder Menschenansammlungen überflogen werden dürfen und welche Mindestabstände einzuhalten sind.

\coleng

Within the Open Category, different drone classes (C0, C1, C2, C3, C4, etc.) are defined, primarily based on maximum take-off weight and technical characteristics. In the context of this handbook, the most relevant classes are C0 (less than 250\,g), C1 (less than 900\,g), and C3 (less than 25\,kg).

\colger

In der Open Category sind verschiedene Drohnenklassen (C0, C1, C2, C3, C4, etc.) definiert, die sich insbesondere anhand des maximalen Abfluggewichts und technischer Eigenschaften unterscheiden. Die im Kontext dieses Handbuchs relevantesten Klassen sind C0 (\(<\)250\,g), C1 (\(<\)900\,g) und C3 (\(<\)25\,kg).

\coleng

Regardless of the drone’s weight, operators must register with the German Federal Aviation Office if the drone is equipped with sensors capable of capturing personal data (e.g., cameras). In addition, liability insurance is mandatory. Depending on the weight and operational scenario, pilots may also be required to hold a remote pilot competency certificate (A1/A3) or an A2 remote pilot certificate.

\colger

Unabhängig vom Gewicht besteht eine Pflicht zur Registrierung des UAS-Betreibers beim Luftfahrt-Bundesamt, sofern die Drohne über Sensoren zur Erfassung personenbezogener Daten (z.\,B. Kameras) verfügt. Zudem ist eine Haftpflichtversicherung gesetzlich vorgeschrieben. Abhängig von Gewicht und Einsatzszenario kann außerdem ein Drohnenführerschein (Kompetenznachweis) A1/A3 oder ein Fernpilotenzeugnis A2 erforderlich sein.

\coleng

Flights in the vicinity of airports, industrial facilities, authorities, or large gatherings of people are either prohibited or subject to authorization. For universities, this often means that outdoor flights are only possible with special permission or on designated areas, whereas flights indoors are not subject to aviation law. Whether operating drones indoors is feasible and appropriate depends on the specific circumstances, including the size and characteristics of both the rooms and the drones used.

\colger

In der Nähe von Flughäfen, Industrieanlagen, Behörden oder Menschenansammlungen sind Flüge ganz oder teilweise untersagt oder genehmigungspflichtig. Für Hochschulen bedeutet dies, dass Außenflüge häufig nur mit Genehmigungen oder auf speziell ausgewiesenen Flächen möglich sind, während Innenräume nicht unter das Luftrecht fallen. Ob der Betrieb von Drohnen in geschlossenen Räumen möglich und sinnvoll ist, hängt jedoch stets vom Einzelfall ab, insbesondere von den räumlichen Gegebenheiten sowie von Größe, Gewicht und Auslegung der eingesetzten Drohnen.

\coleng

In addition, data protection regulations must be observed, in particular the General Data Protection Regulation (GDPR). Video and image recordings may constitute personal data and therefore require appropriate protective measures, transparency obligations, and clearly defined purposes.

\colger

Zusätzlich sind datenschutzrechtliche Vorgaben zu beachten, insbesondere die Datenschutz-Grundverordnung (DSGVO). Video- und Bildaufnahmen können personenbezogene Daten darstellen und erfordern entsprechende Schutzmaßnahmen, Informationspflichten sowie eine klare Zweckbindung.

\colende

\renewcommand{\deutschertitel}{Selbst gebaute Drohnen}
\renewcommand{\englischertitel}{Self-build Drones}
\makrounterabschnitt
\label{AbschnittSelbstgebauteDrohnen}

In contrast to commercially manufactured drones, self-built drones do not have a C-class certification. However, this does not mean that their operation is exempt from regulatory requirements (see Table~\ref{TabelleOpenCategory}) .

\colger

Im Gegensatz zu industriell hergestellten Drohnen verfügen selbst gebaute Drohnen über keine C-Klassifizierung. Dies bedeutet jedoch nicht, dass ihr Betrieb frei von regulatorischen Vorgaben ist (siehe Tabelle~\ref{TabelleOpenCategory}) .

\coleng

Self-built drones with a maximum take-off weight of less than 250\,g generally fall into the Open Category A1. This weight class is subject to the least restrictive regulatory requirements and is therefore particularly attractive for research and teaching. No remote pilot competency certificate is required for drones below 250\,g. If the drone is equipped with sensors capable of capturing personal data (e.g., cameras or microphones), operator registration is mandatory. Liability insurance is required regardless of weight. Flights must be conducted within visual line of sight, and the maximum altitude is limited to 120\,m. The overflight of uninvolved persons and residential, commercial, industrial and recreational areas is permitted, whereas the overflight of assemblies of people is prohibited. No minimum age is specified for pilots of such drones.

\colger

Selbst gebaute Drohnen mit einem Abfluggewicht von weniger als 250\,g fallen grundsätzlich in die Offene Kategorie A1. Diese Gewichtsklasse unterliegt den geringsten regulatorischen Einschränkungen und ist daher besonders attraktiv für Forschung und Lehre. Für Drohnen unter 250\,g ist kein Drohnenführerschein (Kompetenznachweis) erforderlich. Ist die Drohne mit Sensoren zur Erfassung personenbezogener Daten ausgestattet (z.\,B. Kamera oder Mikrofon), ist eine Betreiberregistrierung verpflichtend. Eine Haftpflichtversicherung ist unabhängig vom Gewicht stets erforderlich. Flüge müssen in Sichtweite stattfinden, und die maximale Flughöhe beträgt 120\,m. Der Überflug einzelner unbeteiligter Personen sowie von Wohnsiedlungen, Gewerbegebieten, Industrieanlagen und Freizeitanlagen ist zulässig, der Überflug von Menschenansammlungen hingegen nicht. Für Pilotinnen und Piloten dieser Drohnen ist kein Mindestalter festgelegt.

\coleng

Self-built drones with a take-off weight of 250\,g up to less than 25\,kg fall into the Open Category A3. Operating such drones requires the pilot to hold an A1/A3 competency certificate, which is obtained through an online test. Operator registration with the German Federal Aviation Office and liability insurance are mandatory. A minimum distance of 150\,m from residential, commercial, industrial and recreational areas must be maintained. The overflight of uninvolved persons and assemblies of people is not permitted. A minimum pilot age of 16 years applies.

\colger

Selbst gebaute Drohnen mit einem Abfluggewicht von 250\,g bis unter 25\,kg fallen in die Offene Kategorie A3. Der Betrieb dieser Drohnen erfordert einen Kompetenznachweis A1/A3, der in Form eines Online-Tests erbracht wird. Zudem sind eine Betreiberregistrierung beim Luftfahrt-Bundesamt sowie eine Haftpflichtversicherung verpflichtend. Ein Mindestabstand von 150\,m zu Wohnsiedlungen, Gewerbegebieten, Industrieanlagen und Freizeitanlagen ist einzuhalten. Der Überflug einzelner unbeteiligter Personen sowie von Menschenansammlungen ist nicht erlaubt. Für Pilotinnen und Piloten dieser Drohnen gilt ein Mindestalter von 16 Jahren.

\colende

\begin{table}[htb!]
\centering
\setlength{\tabcolsep}{4pt}       % Default value: 6pt
\renewcommand{\arraystretch}{1.5} % Default value: 1
\begin{threeparttable}
\captionabove{Overview of Drone Classes and Operational Requirements in the Open Category}
\scriptsize
\label{TabelleOpenCategory}
\begin{tabularx}{\textwidth}{X|l|X|l|X}
\toprule
Subcategory & C-Class & Permitted operation & Registration &  Qualification \\
\midrule
A1 -- Overflight of people
& \cellcolor{green}Self-build ($<250\,\mathrm{g}$) 
& Overflight of people allowed but not assemblies of people
& Yes\textsuperscript{a}
& None \\
\cline{2-5}
& C0 ($<250\,\mathrm{g}$) 
& Overflight of people allowed but not assemblies of people
& Yes\textsuperscript{a}
& None \\
\cline{2-5}
& C1 ($<900\,\mathrm{g}$) 
& Overflight of people allowed but not assemblies of people
& Yes
& Online training and online exam \\
\midrule
A2 -- Flying close to people
& C2 ($<4\,\mathrm{kg}$) 
& 30\,m safety distance to uninvolved people (5\,m in low-speed mode)
& Yes
& Online training and online exam, self-study and on-site theory exam \\
\midrule
A3 -- Flying far from people 
& C3/C4 ($<25\,\mathrm{kg}$)
& No uninvolved people endangered, minimum distance of 150\,m from residential, commercial, industrial and recreational areas 
& Yes
& Online training and online exam \\
\cline{2-5}
& \cellcolor{green}Self-build ($<25\,\mathrm{kg}$)
& No uninvolved people endangered, minimum distance of 150\,m from residential, commercial, industrial and recreational areas 
& Yes
& Online training and online exam \\
\bottomrule
\end{tabularx}
\begin{tablenotes}
\footnotesize
\item[a] Registration is required if the drone is equipped with a camera or other personal-data-capturing sensors
\end{tablenotes}
\end{threeparttable}
\end{table}

\FloatBarrier


\renewcommand{\deutschertitel}{Fernpiloten-ID und die UAS-Betreiber-Nummer erwerben}
\renewcommand{\englischertitel}{Obtaining the Remote Pilot ID and the UAS Operator Number}
\makrounterabschnitt
\label{AbschnittKompetenznachweisBetreiberNummer}

The remote pilot ID is obtained via the German Federal Aviation Office’s online portal at \href{https://exam.lba-openuav.de}{exam.lba-openuav.de}. The process begins with a personal registration, during which basic information such as name, address, and email address is provided. After registration, access is granted to the learning and examination area, where the content relevant to the A1/A3 competency certificate is made available. The examination is conducted entirely online as a multiple-choice test. Upon successful completion and uploading a proof of insurance the remote pilot ID is issued and serves as an individual proof of qualification for operating drones in accordance with the applicable regulations.

\colger

Die Fernpiloten-ID wird über das Online-Portal des Luftfahrt-Bundesamts unter \href{https://exam.lba-openuav.de}{exam.lba-openuav.de} beantragt. Zunächst erfolgt eine persönliche Registrierung mit grundlegenden Angaben wie Name, Anschrift und E-Mail-Adresse. Anschließend erhält der Antragsteller Zugriff auf den Lern- und Prüfungsbereich, in dem die für den Kompetenznachweis A1/A3 relevanten Inhalte bereitgestellt werden. Die Prüfung wird vollständig online in Form eines Multiple-Choice-Tests durchgeführt. Nach erfolgreichem Bestehen und der Bereitstellung eines Versicherungsnachweises wird die Fernpiloten-ID im System vergeben und steht als personenbezogener Qualifikationsnachweis für den rechtmäßigen Betrieb von Drohnen in den entsprechenden Kategorien zur Verfügung.

\coleng

The UAS operator number, also referred to as the e-ID, is obtained through the same portal. This requires prior registration as a UAS operator with the German Federal Aviation Office, including the submission of operator and contact information. Once the registration process is completed, a unique e-ID is assigned to the operator. This identifier is used to link the unmanned aircraft system to its registered operator and is mandatory for the lawful operation of drones subject to registration requirements.

\colger

Die UAS-Betreiber-Nummer, auch e-ID genannt, wird über dasselbe Portal beantragt. Voraussetzung ist die Registrierung als UAS-Betreiber beim Luftfahrt-Bundesamt, bei der Angaben zum Betreiber und zu den Kontaktdaten hinterlegt werden. Nach Abschluss der Registrierung wird dem Betreiber eine eindeutige e-ID zugewiesen, die der Identifikation des unbemannten Luftfahrtsystems dient. Diese Kennnummer ist für den Betrieb registrierungspflichtiger Drohnen erforderlich und muss entsprechend den gesetzlichen Vorgaben am UAS angebracht bzw. elektronisch übermittelt werden.

\colende
