\documentclass[
  paper=a4,                % Papierformat A5 (oder benutze "paperwidth" und "paperheight" für benutzerdefinierte Formate)
  DIV=calc,                % Automatische Berechnung des Satzspiegels
  BCOR=0mm,               % Bindekorrektur (z. B. 10 mm, falls notwendig)
  pagesize,                % Seitenformat in die PDF-Metadaten einfügen
  headinclude=true,        % Kolumnentitel einbeziehen
  footinclude=false,        % Fußzeile nicht einbeziehen
]{scrbook}

% Das "cleardoublepage=current" definiert, dass die Vakantseiten am Ende eines Kapitels
% die gleichen Kopf und Fußzeilen haben wie bisher und nicht komplett leer sind!



% Feineinstellungen des Satzspiegels
\KOMAoptions{
  DIV=11,                  % Anpassung des Satzspiegels (höhere Werte = kleinere Ränder)
  fontsize=11pt,           % Schriftgröße (optional, falls Einfluss auf den Satzspiegel gewünscht)
  twoside=true,            % Für zweiseitigen Druck
}


% Mit Fußzeile
% \usepackage[headtopline,headsepline]{scrlayer-scrpage}
% Ohne Fußzeile
\usepackage[headtopline]{scrlayer-scrpage}

\pdfminorversion=7

\RequirePackage{graphicx}
% Die zuletzte angeben Sprache ist die zu Beginn bereitgestellte Sprache. 
% Will man innerhalb des Dokumentes eine deutschsprachige Bezeichnung haben, 
% wie zum Beispiel Inhaltsverzeichnis anstellen von table of contents, 
% muss german beziehungsweise ngerman an der letzten Stelle stehen. 
% Alternativ zu Beginn des Dokumentes die Sprache auf deutsch umstellen:
% \begin{document}\selectlanguage{german}
\usepackage[ngerman,english]{babel} 
% \RequirePackage{english}
\usepackage[T1]{fontenc}
% \usepackage[latin1]{inputenc}
\usepackage[utf8]{inputenc}
\usepackage[tbtags,intlimits]{amsmath}
\usepackage{amssymb}
\usepackage{tabularx}
\usepackage{threeparttable}
\usepackage{pdflscape}   % \begin{landscape} ... \end{landscape}
\usepackage{textcomp}
\usepackage{lmodern}
\usepackage{placeins}   % Für \FloatBarrier. Reihenfolge der Bilder und Tabellen bleibt so absolut stabil. Keine falschen Platzierungen mehr.
\usepackage[official]{eurosym}% Eurosymbol: \euro{}
\usepackage{icomma} % Das verhindert einen großen Abstand nach einem Komma in Formeln.
% Quelle: https://texfragen.de/komma_in_formel


% Use the etoolbox package to patch \thebibliography to use \addcontentsline:
% https://tex.stackexchange.com/questions/71129/bibliography-in-table-of-contents
\usepackage{etoolbox}


%% Seitenlayout
\usepackage[
    a4paper,
    twoside,
    left=25mm,
    right=25mm,
    top=30mm,
    bottom=25mm,
    bindingoffset=0mm
]{geometry}

\usepackage{booktabs}   % Das braucht man u.a. in Tabellen für \toprule, \midrule, \bottomrule
\usepackage{colortbl}   % Das braucht man u.a. in Tabellen für \cellcolor{lightgray} 
\usepackage{tocloft}    % Das braucht man u.a. für \newlistof (deutsches TOC)
\usepackage{paracol}
\usepackage{fancyvrb}  % Das braucht man für die Umgebung Verbatim
\usepackage{needspace}
\usepackage{longtable}
\usepackage{float}      % Für den Positionsierungsparameter H
\usepackage{verbatimbox} % Center verbatim
\usepackage[protrusion=true,expansion=true,spacing=false,kerning=true,tracking=true]{microtype}
% Verbesserter Randausgleich (wichtig bei 2-Spalten-Formatierung!)
% \usepackage{pdfcolparallel} % 2-Spalten-Formatierung ermöglichen!
\usepackage{textcase}
\usepackage{calc}%
% \RequirePackage[bottom]{footmisc}
\usepackage{ragged2e}%
\usepackage{caption}
\captionsetup{
  format=plain,
  labelfont=bf,
  font=small,
  justification=justified,
  singlelinecheck=false
}
\usepackage[leftcaption,raggedright]{sidecap}
\RequirePackage[numbers]{natbib}
% \RequirePackage{multicol}
\usepackage{multirow}
\usepackage{makeidx}
\usepackage{framed}%
\usepackage{url}%
\usepackage{setspace}
\usepackage{listings}
\usepackage{xcolor}
\usepackage{out}
\usepackage{imakeidx}
\usepackage{eso-pic}
\usepackage{rotating} % Für \begin{sideways} ... 
\usepackage{spreadtab,booktabs,numprint}  % Für spreadtab Tabellen
\npthousandsep{,}\npdecimalsign{.} % Zahlenformat

%Das ist wichtig für die Synchronisierung der Seitennummern mit dem PDF-Viewer
\usepackage[hidelinks]{hyperref}

% \setlength{\textwidth}{10.8cm}
% \setlength{\textheight}{15.7cm}

\setlength{\textheight}{\textheight+1cm}


\graphicspath{{./images/}}


% \setlength{\parskip}{0ex plus 10mm minus 0.0ex}
\let\upmu\umu
\let\nohline\relax
\sloppy
\hbadness=10000      % Unterdrückt Underfull \hbox bis Badness 10000
\vbadness=10000      % Unterdrückt Underfull \vbox
\emergencystretch=2em % Das gibt LaTeX etwas “Notfall-Dehnung” => weniger Underfulls, ohne sichtbar anders zu setzen.

\newcommand{\colger}{%
\switchcolumn%
\selectlanguage{ngerman}%
}

\newcommand{\coleng}{%
\switchcolumn*%
\selectlanguage{english}%
}

\newcommand{\colende}{%
\end{paracol}%
}

\newcommand{\colstart}{%
\begin{paracol}{2}[]%
\selectlanguage{english}%
}

\newcolumntype{Y}{>{\centering\arraybackslash}X}%

%%% Trennungsregeln für das gesamte Dokument.
%
\hyphenation{
Figure
address
example
However
however
methods
}

% \selectlanguage{english}

\makeindex[name=de,title={Stichwortverzeichnis}]
\makeindex


\newcommand{\makroabschnitt}{
\Needspace{3.5cm}
\section[\englischertitel]{}
\abschnitt{\deutschertitel}
\begin{paracol}{2}[]
\section*{\englischertitel}
\switchcolumn
\section*{\deutschertitel}
\switchcolumn*
\selectlanguage{english}
}

\newcommand{\makrounterabschnitt}{
\Needspace{2.8cm}
\subsection[\englischertitel]{}
\unterabschnitt{\deutschertitel}
\begin{paracol}{2}[]
\subsection*{\englischertitel}
\switchcolumn
\subsection*{\deutschertitel}
\switchcolumn*
\selectlanguage{english}
}


\newcommand{\makrounterunterabschnitt}{
\Needspace{1.5cm}
\begin{paracol}{2}[]
\subsubsection*{\englischertitel}
\switchcolumn
\subsubsection*{\deutschertitel}
\switchcolumn*
\selectlanguage{english}
}



\newcommand{\nombreindice}{Inhaltsverzeichnis}
\newlistof{inhaltsverzeichnis}{deutschestoc}{\nombreindice}

% Die Tilde ist der Abstand zwischen Zahl und Titel im deutschen Inhaltsverzeichnis
% Das vspace fügt im deutschen Inhaltsverzeichnis einen kleinen Abstand unter jedem Kapiteleintrag ein.
% Beim englischen Inhaltsverzeichnis ist es in Kapitel-Datei im \chapter[...
\newcommand\kapitel[1]{%
  \addcontentsline{deutschestoc}{chapter}{\protect{\vspace{2pt}\thechapter}~#1}%
}
\newcommand\abschnitt[1]{%
\addcontentsline{deutschestoc}{section}{{\thesection}~#1}}
\newcommand\unterabschnitt[1]{%
\addcontentsline{deutschestoc}{subsection}{\protect{\thesubsection}~#1}}

% Die Tilde ist der Abstand zwischen Zahl und Titel im englischen Inhaltsverzeichnis
\renewcommand{\numberline}[1]{#1~}


\definecolor{Gray}{gray}{0.9}

\newcommand{\deutschertitel}{}
\newcommand{\englischertitel}{}

% Hurenkinder und Schusterjungen verhindern
\clubpenalty10000
\widowpenalty10000
\displaywidowpenalty=10000
\interlinepenalty=1000

\pagestyle{scrheadings}

\clearpairofpagestyles
\automark[section]{chapter}

% \pagemark = Seitennummer
% \headmark = Kolumnentitel wie Chapter, Section oder Subsection

% \ohead{Kopfzeile außen}
\ohead{\pagemark}
% \chead{Kopfzeile Mitte}
\chead{}
% \ihead{Kopfzeile innen}
\ihead{\headmark}
% \ifoot{Fußzeile innen}
\ifoot{}
% \cfoot{Fußzeile Mitte}
\cfoot{}
% \ofoot{Fußzeile außen}
\ofoot{}


\renewcommand{\chapterpagestyle}{scrheadings}

\begin{document}

% \newcommand{\heute}{2.~September 2019}
\newcommand{\heute}{\today}

% \AddToShipoutPictureBG{%
%   \AtPageUpperLeft{\raisebox{-1cm}{\makebox[\paperwidth]{\huge \texttt{===>} Entwurf vom \heute\ \texttt{<===} }}}%
% %   \AtPageCenter{\rotatebox{52}{\makebox[0pt]{\Huge This is diagonally across the page}}}
% \AtPageLowerLeft{\raisebox{\baselineskip}{\makebox[\paperwidth]{\huge \texttt{===>}  Entwurf vom \heute\ \texttt{<===}}}}
% }

\frontmatter 
% Inhalt der Titelseite
\begin{titlepage}
\thispagestyle{empty}
\centering
  
{\Huge\bfseries AI Drones\par}

\vspace{1cm}

{\Large\bfseries A Bilingual Handbook on the Design, Construction, and Use of AI-Enabled Drones for Scientific and Educational Projects\par}

\vspace{3cm}

{\Huge\bfseries KI-Drohnen\par}

\vspace{1cm}

{\Large\bfseries Ein zweisprachiges Handbuch über Entwurf, Bau und Einsatz von KI-fähigen Drohnen für wissenschaftliche Projekte und Lehre\par}

\vspace{3cm}

{\large Christian Baun, Theodor Bloch, Matthias Deegener, Oliver Hahm, Martin Kappes, Nur Uddun Syeed\par}
  
\vspace{1cm}
  
{\large Frankfurt University of Applied Sciences}
  
\vfill  
  
{\large \today\par}
\end{titlepage}


% \publishers{Frankfurt University of Applied Sciences}

\include{chapter00}  % preface
\cleardoublepage
\tableofcontents
\cleardoublepage
\listofinhaltsverzeichnis
\cleardoublepage
% \markboth{}{}

\mainmatter 
\ihead{\englischertitel} % Ohne diese manuelle Zuweisung ist die Kopfzeile in den folgenden Kapiteln fehlerhaft
\renewcommand{\deutschertitel}{Hardware}
\renewcommand{\englischertitel}{Hardware}

% !!! Das hier war vorher !!!
% \chapter[\englischertitel]{\englischertitel\newline\deutschertitel}
% !!! Das hier war vorher !!!


% Das vspace fügt im Inhaltsverzeichnis einen kleinen Abstand unter dem Kapiteleintrag ein.
% Beim deutschen Inhaltsverzeichnis ist es im book.tex an nur einer Stelle in der Zeile 
% \addcontentsline{deutschestoc}{chapter}{\protect{\vspace{2pt}\thechapter}~#1}}
\chapter[\protect{\vspace{2pt}\englischertitel}]{}
\kapitel{\deutschertitel}

\label{KapitelHardware}

\begin{paracol}{2}[]

{\raggedright\huge\bfseries\sffamily \englischertitel \par\ } \\[1.8ex]

\switchcolumn

{\raggedright\huge\bfseries\sffamily \deutschertitel \par\ } \\[1.8ex]

\switchcolumn*
\selectlanguage{english}

TBD

\switchcolumn
\selectlanguage{ngerman}

TBD

\end{paracol}

\renewcommand{\deutschertitel}{Rahmen}
\renewcommand{\englischertitel}{Frames}

\makroabschnitt

\label{AbschnittFrames}

TBD

\switchcolumn
\selectlanguage{ngerman}

Der Rahmen aus verbindet alle Komponenten der Drohne. Das verwendete Material ist üblicherweise Carbon. Dabei handelt es sich um einen leichtgewichtigen und dennoch hochfesten und verwindungssteifen Verbundwerkstoff aus Kohlenstofffasern. Seltener kommen auch Rahmen aus Kunststoff zum Einsatz. Der Rahmen definiert die Propellergröße (siehe Abschnitt~\ref{AbschnittPropeller}).

\switchcolumn*
\selectlanguage{english}

TBD

\switchcolumn
\selectlanguage{ngerman}

Der Rahmen nimmt üblicherweise zentral die wichtigsten elektronischen Komponenten wie Flugcontroller, Videosender, Empfänger und Kamera auf, um diese zu schützen. Der Akku befindet sich üblicherweise oben auf der Drohne, um Beschädigungen beim Landen zu vermeiden. 

\switchcolumn*
\selectlanguage{english}

TBD

\switchcolumn
\selectlanguage{ngerman}

Wichtige Unterscheidungskriterien bei der Auswahl des passenden Rahmens sind auch die Abstände der Bohrlöcher zur Befestigung des Flugcontrollers und des Videosenders. Gängige Maße sind: 

\switchcolumn*
\selectlanguage{english}

TBD

\switchcolumn
\selectlanguage{ngerman}

\begin{itemize}
\item 30,5 x 30,5\,mm
\item 25,5 x 25,5\,mm
\item 20 x 20\,mm
\end{itemize}


\switchcolumn*
\selectlanguage{english}

TBD

\switchcolumn
\selectlanguage{ngerman}

Verfügt ein Rahmen nicht über passende Bohrlöcher für den ausgewählten Flugcontrollers und den Videosender, kann ein per 3D-Drucker gedruckte Adapter helfen, wenn der Platz im Rahmen dafür ausreicht. 



\end{paracol}






\renewcommand{\deutschertitel}{Flugcontroller}
\renewcommand{\englischertitel}{Flight Controller}

\makroabschnitt

\label{AbschnittFC}

TBD

\switchcolumn
\selectlanguage{ngerman}

TBD

\end{paracol}



\renewcommand{\deutschertitel}{Videosender}
\renewcommand{\englischertitel}{Video Transmitter}

\makroabschnitt

\label{AbschnittVTX}

TBD

\switchcolumn
\selectlanguage{ngerman}

TBD

\end{paracol}




\renewcommand{\deutschertitel}{Propeller}
\renewcommand{\englischertitel}{Propeller}

\makroabschnitt

\label{AbschnittPropeller}

TBD

\switchcolumn
\selectlanguage{ngerman}

TBD

\end{paracol}

\renewcommand{\deutschertitel}{Software}
\renewcommand{\englischertitel}{Software}

% !!! Das hier war vorher !!!
% \chapter[\englischertitel]{\englischertitel\newline\deutschertitel}
% !!! Das hier war vorher !!!


% Das vspace fügt im Inhaltsverzeichnis einen kleinen Abstand unter dem Kapiteleintrag ein.
% Beim deutschen Inhaltsverzeichnis ist es im book.tex an nur einer Stelle in der Zeile 
% \addcontentsline{deutschestoc}{chapter}{\protect{\vspace{2pt}\thechapter}~#1}}
\chapter[\protect{\vspace{2pt}\englischertitel}]{}
\kapitel{\deutschertitel}

\label{KapitelSoftware}

\begin{paracol}{2}[]

{\raggedright\huge\bfseries\sffamily \englischertitel \par\ } \\[1.8ex]

\switchcolumn

{\raggedright\huge\bfseries\sffamily \deutschertitel \par\ } \\[1.8ex]

\coleng

TBD

\colger

TBD

\colend

\renewcommand{\deutschertitel}{Flight Controller Firmware}
\renewcommand{\englischertitel}{Flight Controller Firmware}
\makroabschnitt
\label{AbschnittFirmwareFC}

TBD

\colger

TBD


\coleng

TBD

\colger

TBD

\colend

\renewcommand{\deutschertitel}{Betaflight}
\renewcommand{\englischertitel}{Betaflight}
\makrounterabschnitt
\label{AbschnittBetaflight}

TBD

\colger

TBD

\colend

\renewcommand{\deutschertitel}{INAV}
\renewcommand{\englischertitel}{INAV}
\makrounterabschnitt
\label{AbschnittINAV}

TBD

\colger

TBD

\colend

\renewcommand{\deutschertitel}{ArduPilot}
\renewcommand{\englischertitel}{ArduPilot}
\makrounterabschnitt
\label{AbschnittArduPilot}

TBD

\colger

TBD

\colend


\renewcommand{\deutschertitel}{Fernbedienung Firmware}
\renewcommand{\englischertitel}{Remote Control Firmware}
\makroabschnitt
\label{AbschnittFirmwareRemoteControl}

TBD

\colger

TBD


\coleng

TBD

\colger

TBD

\colend

\renewcommand{\deutschertitel}{EdgeTX}
\renewcommand{\englischertitel}{EdgeTX}
\makrounterabschnitt
\label{AbschnittEdgeTX}

TBD

\colger

TBD

\colend



\renewcommand{\deutschertitel}{Sendemodul- und Empfänger-Firmware}
\renewcommand{\englischertitel}{Transmitter and Receiver Firmware}
\makroabschnitt
\label{AbschnittFirmwareReceiverTransmitter}

TBD

\colger

TBD


\coleng

TBD

\colger

TBD

\colend

\renewcommand{\deutschertitel}{ExpressLRS}
\renewcommand{\englischertitel}{ExpressLRS}
\makrounterabschnitt
\label{AbschnittExpressLRS}

TBD

\colger

TBD

\colend


\renewcommand{\deutschertitel}{KI-Drohnen-Projekte}
\renewcommand{\englischertitel}{AI Drone Projects}

% !!! Das hier war vorher !!!
% \chapter[\englischertitel]{\englischertitel\newline\deutschertitel}
% !!! Das hier war vorher !!!


% Das vspace fügt im Inhaltsverzeichnis einen kleinen Abstand unter dem Kapiteleintrag ein.
% Beim deutschen Inhaltsverzeichnis ist es im book.tex an nur einer Stelle in der Zeile 
% \addcontentsline{deutschestoc}{chapter}{\protect{\vspace{2pt}\thechapter}~#1}}
\chapter[\protect{\vspace{2pt}\englischertitel}]{}
\kapitel{\deutschertitel}

\label{KapitelKI}

\begin{paracol}{2}[]

{\raggedright\huge\bfseries\sffamily \englischertitel \par\ } \\[1.8ex]

\switchcolumn

{\raggedright\huge\bfseries\sffamily \deutschertitel \par\ } \\[1.8ex]

\coleng

TBD

\colger

Dieses Kapitel beschreibt KI-Anwendungen, die FPV-Drohnen zur Datenerfassung und/oder als Tarnsportvehikel nutzen. Vorgestellt werden Anwendungen, die in Forschungsprojekten, Lehrveranstaltungen und Abschlussarbeiten in der Lehreinheit Informatik an der Frankfurt University of Applied Sciences entwickelt, implementiert und evaluiert wurde. Das Kapitel beschreibt zu jeder dieser KI-Anwendungen die benötigten zusätzlichen Hard- und Softwarekomponenten, nötige Schritte zur Realisierung und die Kosten.

\colende

\renewcommand{\deutschertitel}{Objekterkennung}
\renewcommand{\englischertitel}{Object Detection}
\makroabschnitt
\label{AbschnittObjekterkennung}

TBD

\colger

TBD

\colende


\renewcommand{\deutschertitel}{Autopilot}
\renewcommand{\englischertitel}{Auto Pilot}
\makroabschnitt
\label{AbschnittAutopilot}

TBD

\colger

TBD

\colende



\renewcommand{\deutschertitel}{Follow-Me-Function}
\renewcommand{\englischertitel}{Follow Me Function}
\makroabschnitt
\label{AbschnittAutopilot}

TBD

\colger

TBD

\colende

\include{chapter04}
\renewcommand{\deutschertitel}{Autopilot}
\renewcommand{\englischertitel}{Autopilot}
\chapter[\protect{\vspace{2pt}\englischertitel}]{}
\kapitel{\deutschertitel}

\label{KapitelAutopilot}

\begin{paracol}{2}[]

{\raggedright\huge\bfseries\sffamily \englischertitel \par\ } \\[1.8ex]

\switchcolumn

{\raggedright\huge\bfseries\sffamily \deutschertitel \par\ } \\[1.8ex]

\coleng

An autopilot system enables an FPV drone to perform autonomous or semi-autonomous tasks such as position hold, altitude hold, waypoint missions, or return-to-home procedures. These functions rely on a combination of hardware components, sensors, and flight-controller firmware that interprets sensor data and executes control commands. 

\colger

Ein Autopilot-System ermöglicht es einer FPV-Drohne, autonome oder teilautonome Aufgaben wie das Halten der Position, das Halten der Flughöhe, Wegpunktmissionen oder die Rückkehr zum Startpunkt auszuführen. Diese Funktionen basieren auf einer Kombination aus Hardwarekomponenten, Sensoren und Flugcontroller-Firmware, welche die Sensordaten auswertet und entsprechende Steuerbefehle umsetzt. 

\coleng

Various firmware platforms provide differing levels of capability and complexity, including Betaflight, iNAV, ArduPilot, and PX4. The following sections outline the requirements, configuration steps, and supported functions for each of these systems.

\colger

Verschiedene Firmware-Plattformen bieten hierfür unterschiedliche Fähigkeiten und Komplexitätsstufen, darunter Betaflight, iNAV, ArduPilot und PX4. Die folgenden Abschnitte beschreiben die Anforderungen, Konfigurationsschritte und unterstützten Funktionen der jeweiligen Systeme.

\colende

\renewcommand{\deutschertitel}{Autopilot mit Betaflight}
\renewcommand{\englischertitel}{Autopilot with Betaflight}
\makroabschnitt
\label{AbschnittAutopilotBetaflight}

Betaflight (see Section~\ref{AbschnittBetaflight}) is primarily designed for manual flight (freestyle and racing) and does not provide a full-featured autopilot. However, limited autonomous functions can be implemented, such as angle-stabilized flight, horizon-assisted stabilization, and basic failsafe procedures. 

\colger

Betaflight (siehe Abschnitt~\ref{AbschnittBetaflight}) ist primär für manuelles Fliegen (Freestyle und Rennen) ausgelegt und stellt keinen vollwertigen Autopiloten bereit. Es lassen sich jedoch begrenzte autonome Funktionen realisieren, etwa stabilisierte Flugmodi (Angle, Horizon) oder grundlegende Failsafe-Verfahren. 

\coleng

To achieve these functions, the flight controller requires an accelerometer, a gyroscope, and optionally a barometer. GPS support is available only for basic features such as rescue mode. Betaflight does not support waypoint missions or advanced navigation. 

\colger

Für diese Funktionen werden ein Beschleunigungssensor, ein Gyroskop und optional ein Barometer benötigt. GPS-Unterstützung existiert nur in eingeschränkter Form, beispielsweise für den Rescue Mode. Wegpunktmissionen oder fortgeschrittene Navigationsaufgaben werden nicht unterstützt. 

\coleng

Configuration is performed through the Betaflight Configurator or via the Betaflight web application, where stabilization modes, failsafe behavior, and optional rescue functionality can be assigned to AUX channels.

\colger

Die Konfiguration erfolgt über den Betaflight Configurator oder über die Betaflight Webanwendung, wo Stabilisierung, Failsafe-Verhalten und optional Rettungsfunktionen auf AUX-Kanäle gelegt werden können.

\colende

\renewcommand{\deutschertitel}{Autopilot mit iNAV}
\renewcommand{\englischertitel}{Autopilot with iNAV}
\makroabschnitt
\label{AbschnittAutopilotINAV}

The iNAV firmware (see Section~\ref{AbschnittINAV}) implements full GPS-assisted flight and is designed for navigation, position hold, altitude hold, return-to-home, and basic autnomous waypoint missions.

\colger

Die Firmware iNAV (siehe Abschnitt~\ref{AbschnittINAV}) implementiert eine vollständige GPS-unterstützte Flugsteuerung bereit und eignet sich für Navigation, Position Hold, Altitude Hold, Return-to-Home und grundlegende autonome Wegpunktmissionen.

\coleng

Supported hardware includes flight controllers with gyroscope, accelerometer, barometer, magnetic compass, and a GPS module. Also airspeed sensors are supported by iNAV.

\colger

Hierfür ist es erforderlich, dass der Beschleunigungssensor und die GPS-Antenne konfiguriert sind.
Außerdem muss ein Barometer für die Höhenmessung vorhanden sein. Ein Kompass (Magnetometer) ist nicht zwingend erforderlich ab INAV-Version 7.1, jedoch dringend empfohlen für bessere Performance.

Insofern ein Kompass vorhanden ist, sollte dieser korrekt ausgerichtet und kalibriert werden sowie idealerweise mindestens 10 cm von den Motoren und anderen elektrischen Komponenten entfernt sein. Auch Fluggeschwindigkeitssensoren werden von INAV unterstützt.
 
\coleng

The iNAV Configurator is used to configure sensors, calibrate the magnetic compass, specify mixer settings, and assign flight modes. iNAV enables basic autonomous missions but is not as feature-rich as ArduPilot or PX4.

\colger

Zur Konfiguration der Autopilotfunktionalitäten genügt der INAV Configurator. In diesem können sämtliche Flugmodi und Tuningeinstellungen getätigt werden. Nicht alle der folgend erläuterten Modi sind vollwertige Flugmodi. Einige verändern lediglich Eigenschaften des Flugverhaltens und müssen in Kombination mit anderen Modi verwendet werden.

\begin{description}
	\item[NAV ALTHOLD] ist ein solcher Modus. Dieser setzt lediglich die Flughöhe fest und muss in Kombination mit einem manuellen Flugmodus verwendet werden. Hierbei ist ANGLE empfohlen, da ALTHOLD nicht für Überkopf-Maneuver geeignet ist. Im "Advanced Tuning" Tab kann mit "Max. Alt-hold climb rate" der Spielraum konfiguriert werden, mit welchem der Pilot die Höhe manuell nach oben und unten variieren kann.

	\item[NAV GCS] ist ein weiterer unterstützender Modus. Dieser erlaubt es einer verbundenen Ground-Control-Station (GCS) die Position der Drohne zu kontrollieren. Dies kann für die Follow-Me-Funktion verwendet werden.
\end{description}

\begin{description}
	\item[NAV POSHOLD] ist ein eigenständiger Modus, welcher die Drohne die Position halten lässt. Durch anpassen der Throttle-, Pitch- und Roll-Sticks kann die Position verändert werden. Sobald die jeweiligen Sticks in die Ausgangsposition versetzt werden, hält die Drohne die neue Position.

	\item[MC BRAKING] unterstützt den NAV POSHOLD Modus und wird zusammen mit diesem verwendet. Der Unterschied liegt im Bremsverhalten bei Positionsanpassungen. Beim einfachen POSHOLD kehrt die Drohne nach Abschluss des Bremsvorgangs an den Ort zurück, an dem die Sticks losgelassen wurden. Mit MC BRAKING bremst die Drohne stärker und hält die Position an dem Ort an welchem sie zum Stillstand kommt.

	\item[NAV COURSE HOLD] versucht die Drohne den derzeitigen Kurs zu halten. Die Geschwindigkeit kann in diesem Modus mithilfe des Pitch-Sticks angepasst werden. Der Flugcontroller passt die Geschwindigkeit proportional zur Stellung des Pitch-Sticks an.

	\item[NAV CRUISE] entspricht einer Kombination aus NAV COURSE HOLD und NAV ALTHOLD.

	\item[NAV RTH] ist der Return-To-Home (RTH)-Modus, welcher auch als Failsafe konfiguriert werden kann. Failsafe sollte jedoch nicht für ein kontrolliertes RTH aktiviert werden. Unter Umständen kann es zu unerwartetem Verhalten kommen. Ist die Drohne mehr als 10 Meter von der Arming-Location oder dem konfigurierten Safehome entfernt, klettert sie erst auf eine konfigurierte Höhe und fliegt dann in Richtung Home und landet dort, sobald sie dort im Radius von einem Meter ist. Im Advanced-Tuning-Tab kann das RTH-Verhalten mit vielen Einstellungen weitreichend modifiziert werden.

	\item[HOME RESET] ist ein komplementärer Modus zu NAV RTH. Dieser sollte auf einen RC-Switch gelegt werden, welcher nach dem Drücken auf die Ursprungsposition zurückkehrt. Dadurch wird es ermöglicht während des Fluges die aktuelle Safehome-Location zu überschreiben. Standardmäßig ist die letzte Arming-Location das Safehome, jedoch können auch bis zu acht unterschiedliche Safehomes manuell konfiguriert werden, wovon dann jeweils das nächste angeflogen wird, wenn sich dieses innerhalb der maximalen Distanz befindet, die ein solches Safehome entfernt sein darf.

	\item[NAV WP] erlaubt es der Drohne, autonom an einer vordefinierten Sequenz von Wegpunkten entlangzufliegen. Diese Wegpunkte können entweder manuell vor Ablug definiert werden, beispielsweise über den Mission Control Tab im INAV Configurator oder mithilfe des Modus WP PLANNER während dem manuellen Flug.

	\item[WP PLANNER] ist kein eigenständiger Flugmodus, sondern dient dazu, Wegpunkte während des manuellen Flugs zu definieren. Wichtig ist, dass WP nicht aktiviert sein darf, sonst kann kein Wegpunkt gespeichert werden. Darüber hinaus muss der Block MISSION INFO im OSD aktiviert werden. Dieser Block zeigt die für den Planner notwendigen Informationen an. Um einen Wegpunkt zu speichern, muss die Drohne an den gewünschten Ort bewegt und der Switch, auf dem WP PLANNER konfiguriert ist, betätigt werden. Sobald der Speichervorgang abgeschlossen ist, wird eine entsprechende Mitteilung im OSD zu sehen sein. Beim Speichern ist zu beachten, dass die Höhe entsprechend mitgespeichert wird. Dies ist wichtig, da ein Wegpunkt, der ebenerdig gespeichert wurde, entsprechend angeflogen wird. Um die so konfigurierte Mission zu fliegen, genügt es PLANNER zu deaktivieren und WP zu aktivieren. Alternative kann die Mission auch mit Stick-Commands gespeichert werden (siehe Kapitel xxx).

\end{description}

 \colende

\renewcommand{\deutschertitel}{Wegpunktmissionen mit INAV}
\renewcommand{\englischertitel}{Waypoint Missions with INAV}
\makrounterabschnitt
\label{INAVWaypoint}

Test

\colger

Dieses Unterkapitel dient dazu, die Vorbereitungen einer Wegpunktmission umfassender zu beschreiben. Nachdem der Modus auf einen entsprechenden Switch konfiguriert ist, sind noch einige Optionen im Advanced-Tuning-Tab unter den Reitern Multirotor Navigaton Settings, General Navigation Settings und Waypoint Navigation Settings zu konfigurieren. Was diese Optionen bewirken, wird im Reiter selbst erläutert und hier nicht noch einmal ausgeführt.

Nach der Konfiguration des Advanced Tuning Tabs kommt es nun zum eigentlichen Setzen der Wegpunkte. Die Wegpunkte können durch einfaches Klicken auf der Karte gesetzt und aneinandergereiht werden.

\colende

\begin{figure}[htb!]
	\centering
	\includegraphics[width=\linewidth]{INAV_Mission_Control_Demo_Example.png}
	\caption{Example of a waypoint mission}
	\label{INAV_Mission_Control_Demo_Example}
\end{figure}

\colstart

Bumms - auf der Fahrbahn...

\colger

Durch erneutes Klicken auf einen jeweiligen Wegpunkt kann dieser editiert werden. Längen- und Breitengrade können über das Nummernfeld verfeinert werden und es ist darüber hinaus möglich, den Wegpunkttypen zu konfigurieren und Actions zu definieren.

Beim Wegpunkttypen besteht die Auswahl zwischen Waypoint, Position-Hold (PH), Point-of-Interest (POI) und Land. Der Waypoint beschreibt einen einfachen Wegpunkt, durch den die Drohne hindurch fliegt. PH beschreibt einen Wegpunkt an dem die Drohne für eine bestimmte Zeit anhält, welche im Feld Wait time konfiguriert wird. POI ist ein Punkt in der Mission an dem sich die Ausrichtung der Drohne im Flug orientiert. Wird ein Wegpunkt als POI definiert, richtet sich die Kamera nach diesem, anstatt sich am Weg auszurichten, während sie zum nächsten Wegpunkt fliegt. Land lässt die Drohne landen, sobald der Punkt erreicht wurde, anstatt die Mission fortzuführen.

Bei den Actions hat man JUMP, SET HEAD und RTH zur Verfügung. Mit Jump kann zu einem anderen Wegpunkt gesprungen werden, als dem mit der nächstgrößeren Nummer. In P1 wird die ID des anzufliegenden Wegpunkts eingetragen und in P2 die Anzahl der Jump-Wiederholungen.

 Mit SET HEAD kann die Ausrichtung der Drohne gesteuert werden. Dazur wird die entsprechende Gradzahl zwischen 0 und 359 in das Feld P1 eingetragen (Kommazahlen sind möglich). Sobald die Drohne den Wegpunkt erreicht hat wird die Kamera entsprechend ausgerichtet für den weiteren Flug.

 RTH lässt die Drohne zum konfigurierten Safehome zurückfliegen, sobald sie am jeweiligen Wegpunkt angekommen ist. Dies eignet sich besonders für das Ende von Missionen.

Ein weiterer wichtiger Punkt ist die jeweilige Höhe, an der ein Wegpunkt angeflogen wird. Hier muss sichergestellt sein, dass die in der EU vorgeschriebene Maximalhöhe von 120 Metern nicht überschritten wird. Außerdem ist darauf zu achten, dass die Drohne an jedem Wegpunkt die entsprechende Höhe hat, gemessen am Umgebungsprofil. Dazu kann die Option Sea Level Ref herangezogen werden, wenn die Höhe konfiguriert wird.

Über den Button MP Elevation am kann das Höhenprofil für die Mission, mit dem Höhenprofil der Umgebung verglichen werden. Damit kann vermieden werden, dass die Drohne während der Mission mit dem Boden kollidiert. Um diesen Vergleich zu ermöglichen, muss jedoch eine Form von Take-off-Location vorhanden werden.

\colende

\begin{figure}[htb!]
	\centering
	\includegraphics[width=\linewidth]{INAV_Mission_Control_Elevation_Profile.png}
	\caption{Mission Control Example Elevation Profile}
	\label{INAV_Mission_Control_Elevation_Profile}
\end{figure}

\colstart

Test

\colger

Ist die Wegpunkt-Mission fertig konfiguriert bestehen insgesamt drei Optionen zum Speichern. Die erste ist das Speichern in eine Datei auf dem Computer und die zweite ist das Speichern in den volatilen Speicher des Flugcontrollers (FC). Letzteres ist nur geeignet, wenn die Mission sofort gefolgen werden soll, da sie bei einem Neustart verloren geht. Soll die Mission dauerhaft gespeichert werden, so eignet sich die dritte Option, das Speichern in den EEPROM des FC.

\colende


\renewcommand{\deutschertitel}{Laden von Missionen und INAV Stickbefehle}
\renewcommand{\englischertitel}{Loading Missions and INAV Stick Commands}
\makrounterabschnitt
\label{INAVStickCommands}

Test

\colger

Insofern die Mission in den EEPROM geschrieben wurde, muss sie in den volatilen Speicher des FC geladen werden. Dies kann ebenfalls im Mission Control Tab erfolgen, über den Button Load from EEPROM. Sollte kein Computer vorhanden sein, kann das automatische Laden einer Mission beim Boot im Advanced Tuning Tab konfiguriert werden.

Alternativ kann die Mission auch mit einem entsprechenden Stick-Command geladen werden. Stick-Commands sind verschiedene Kombinationen der beiden Sticks, welche normalerweise zum Fliegen benutzt werden. Sie können eine Reihe verschiedener Funktionen aktivieren, wie beispielsweise das OSD-Menü. Abbildung X zeigt die Stick-Commands, welche es in INAV gibt.

\colende

\begin{figure}[htb!]
	\centering
	\includegraphics[width=\linewidth]{INAV_Stick_Commands.png}
	\caption{Stick Commands in INAV}
	\label{INAV_Stick_Commands}
	% Source: https://raw.githubusercontent.com/iNavFlight/inav/refs/heads/master/docs/assets/images/StickPositions.png
\end{figure}

\renewcommand{\deutschertitel}{Autopilot mit ArduPilot}
\renewcommand{\englischertitel}{Autopilot with ArduPilot}
\makroabschnitt
\label{AbschnittAutopilotArduPilot}

ArduPilot (see Section~\ref{AbschnittArduPilot}) provides a comprehensive autopilot system for drones. It enables advanced autonomous functions such as waypoint navigation, mission planning, geofencing, object avoidance with additional sensors, automatic takeoff and landing, and precision positioning.

\colger

ArduPilot (siehe Abschnitt~\ref{AbschnittArduPilot}) bietet ein umfassendes Autopilotsystem für Drohnen, Flächenflugzeuge, Hubschrauber und andere Plattformen. Es unterstützt fortgeschrittene autonome Funktionen wie Wegpunktnavigation, Missionsplanung, Geofencing, Hindernisvermeidung mit zusätzlichen Sensoren, automatischen Start und automatische Landung sowie präzise Positionsbestimmung. 

\coleng

Depending on the firmware variant, ArduPilot requires powerful hardware with sufficient flash memory (see Section~\ref{AbschnittFC}) and sensors including barometer, magnetic compass, accelerometer, gyroscope, and GPS. Additional sensors such as optical flow sensors (Light Detection and Ranging -- lidar), or airspeed sensors can be integrated. 

\colger

e nach Firmware-Variante benötigt ArduPilot leistungsfähige Hardware mit ausreichend Flash-Speicher (siehe Abschnitt~\ref{AbschnittFC}) sowie Sensoren wie Barometer, magnetischem Kompass, Beschleunigungssensor, Gyroskop und GPS. Zusätzliche Sensoren wie optische Flusssensoren (Light Detection and Ranging -- Lidar) oder Fluggeschwindigkeitssensoren können integriert werden. 

\coleng

The Mission Planner software is used for configuration, calibration, and mission definition.

\colger

Konfiguration, Kalibrierung und Missionsplanung erfolgen über die Software Mission Planner.

\colende

\renewcommand{\deutschertitel}{Autopilot mit PX4}
\renewcommand{\englischertitel}{Autopilot with PX4}
\makroabschnitt
\label{AbschnittAutopilotPX4}

PX4 is an open-source autopilot platform similar to ArduPilot, designed for research, industry applications, and autonomous robotics. It supports advanced mission planning, companion computers (e.g., Raspberry Pi Compute Module\,4), and complex multi-sensor navigation. 

\colger

PX4 ist eine Open-Source-Autopilotplattform, die ähnlich wie ArduPilot auf Forschungs-, Industrie- und Robotikanwendungen ausgerichtet ist. Die Firmware unterstützt fortgeschrittene Missionsplanung, Companion-Computer (z.B. Raspberry Pi Compute Module\,4), und komplexe multisensorbasierte Navigation. 

\coleng

PX4 requires hardware such as Pixhawk controllers or compatible boards with powerful processors and extensive sensor capability. Configuration and mission planning are performed using the QGroundControl software. 

\colger

PX4 erfordert Hardware wie Pixhawk-Controller oder kompatible Boards mit leistungsfähigen Prozessoren und umfangreicher Sensorunterstützung. Die Konfiguration und Missionsplanung erfolgt über die Software QGroundControl. 

\coleng

Due to its modular design and focus on extensibility, PX4 is frequently used in industrial environments, although some FPV-specific features are less developed compared to Betaflight or iNAV.

\colger

Aufgrund seiner modularen Architektur und Erweiterbarkeit wird PX4 häufig in industriellen Projekten eingesetzt, auch wenn FPV-spezifische Funktionen teilweise weniger ausgeprägt sind als bei Betaflight oder iNAV.

\colende

\renewcommand{\deutschertitel}{Follow-Me-Funktion}
\renewcommand{\englischertitel}{Follow-Me Function}
\chapter[\texorpdfstring{\protect{\vspace{2pt}\englischertitel}}{\englischertitel}]{}
\kapitel{\deutschertitel}
\thispagestyle{empty}

\label{KapitelFollowme}

\begin{paracol}{2}[]

{\raggedright\huge\bfseries\sffamily \englischertitel \par\ } \\[1.8ex]

\switchcolumn

{\raggedright\huge\bfseries\sffamily \deutschertitel \par\ } \\[1.8ex]

\coleng

The Follow-Me function allows an FPV drone to track and follow a moving target, typically a person carrying a GPS device, a smartphone, or a companion transmitter. This capability is useful for recording moving subjects, supporting inspection tasks, or enabling semi-autonomous flight operations. 

\colger

Die Follow-Me-Funktion ermöglicht es einer FPV-Drohne, einem sich bewegenden Ziel zu folgen, typischerweise einer Person, die ein GPS-Gerät, ein Smartphone oder einen zusätzlichen Sender mit sich führt. Diese Fähigkeit wird unter anderem zur Aufnahme bewegter Szenen, zur Unterstützung von Inspektionsaufgaben oder für teilautonome Fluganwendungen genutzt. 

\coleng

Depending on the firmware, the Follow-Me function may rely on external GPS data, telemetry links, onboard sensors, or a combination of these. The following sections describe how Follow-Me is implemented in Betaflight, iNAV, ArduPilot, and provide an outlook on its implementation in PX4. 

\colger

Je nach Firmware basiert die Follow-Me-Funktion auf externen GPS-Daten, Telemetrieverbindungen, an Bord befindlichen Sensoren oder einer Kombination dieser Quellen. Die folgenden Abschnitte beschreiben die Umsetzung der Follow-Me-Funktion in Betaflight, iNAV, ArduPilot sowie einen Ausblick auf die Implementierung in PX4. 

\colende

\renewcommand{\deutschertitel}{Follow-Me mit Betaflight}
\renewcommand{\englischertitel}{Follow-Me with Betaflight}
\makroabschnitt
\label{AbschnittFollowMeBetaflight}

Betaflight does not provide a Follow-Me mode, as the firmware is primarily optimized for manual flight (freestyle and racing). Only basic GPS-assisted functions such as rescue mode are available. Follow-Me operation is not supported because Betaflight lacks autonomous navigation, positional tracking, and target-following logic. Implementing Follow-Me functionality would require a transition to iNAV or ArduPilot.

\colger

Betaflight bietet keinen Follow-Me-Modus, da die Firmware primär für für manuelles Fliegen (Freestyle und Rennen) ausgelegt ist. Lediglich grundlegende GPS-gestützte Funktionen wie der Rescue Mode sind verfügbar. Ein Follow-Me-Betrieb wird nicht unterstützt, da Betaflight keine autonome Navigation, keine Positionsverfolgung und keine Zielverfolgungslogik bereitstellt. Für den Einsatz der Follow-Me-Funktion wird daher ein Wechsel zu iNAV oder ArduPilot erforderlich.

\colende

\renewcommand{\deutschertitel}{Follow-Me mit iNAV}
\renewcommand{\englischertitel}{Follow-Me with iNAV}
\makroabschnitt
\label{AbschnittFollowMeINAV}

iNAV provides a dedicated Follow-Me mode that uses the GPS position of an external device, typically the radio transmitter equipped with an external GPS module or a compatible telemetry system. 

\colger

iNAV stellt einen eigenen Follow-Me-Modus bereit, der die GPS-Position eines externen Geräts verwendet, üblicherweise die Fernsteuerung mit einem angeschlossenen GPS-Modul oder einem kompatiblen Telemetriesystem. Für Follow-Me in iNAV werden ein Flugcontroller mit GPS, Barometer, magnetischen Kompass sowie eine stabile GPS-Empfangsqualität benötigt.

\coleng

Follow-Me in iNAV requires a flight controller with GPS, barometer, magnetic compass, and stable GPS reception. The external device transmits its position to the flight controller via a telemetry protocol such as MAVLink, MSP, or a proprietary transmitter protocol. 

\colger

Das externe Gerät übermittelt seine Position über ein Telemetrieprotokoll wie MAVLink, MSP oder ein proprietäres Fernsteuerungsprotokoll an den Flugcontroller. 

\coleng

Configuration is performed in the iNAV Configurator by enabling the Follow-Me mode, assigning it to an AUX channel, and configuring the GPS update rate and safe distances. iNAV supports various follow patterns, including direct pursuit and offset tracking.

\colger

Die Konfiguration erfolgt im iNAV Configurator durch Aktivieren des Follow-Me-Modus, das Zuweisen eines AUX-Kanals sowie die Einstellung der GPS-Aktualisierungsrate und der Sicherheitsabstände. iNAV unterstützt unterschiedliche Verfolgungsprofile, darunter direkte Verfolgung und seitlich versetztes Tracking.

\colende

\renewcommand{\deutschertitel}{Follow-Me mit ArduPilot}
\renewcommand{\englischertitel}{Follow-Me with ArduPilot}
\makroabschnitt
\label{AbschnittFollowMeArduPilot}

ArduPilot provides a comprehensive Follow-Me implementation that supports external GPS devices, companion computers (e.g., Raspberry Pi Zero, Raspberry Pi Compute Module\,4/5, NVidia TX1/TX2), and smartphone applications via MAVLink. Follow-Me requires a GPS-equipped external device. The drone must include a barometer, magnetic compass, accelerometer, gyroscope, and GPS module. 

\colger

ArduPilot bietet eine umfassende Follow-Me-Implementierung, die externe GPS-Geräte, Companion-Computer (z.B. Raspberry Pi Zero, Raspberry Pi Compute Module\,4/5, NVidia TX1/TX2) und Smartphone-Anwendungen über MAVLink unterstützt. Für Follow-Me wird ein externes GPS-Gerät benötigt. Die Drohne muss über Barometer, magnetischen Kompass, Beschleunigungssensor, Gyroskop und ein GPS-Modul verfügen. 

\coleng

Follow-Me is enabled in Mission Planner by selecting the Follow-Me feature and configuring parameters such as follow distance, altitude offset, and update frequency. 

\colger

Die Aktivierung erfolgt im Mission Planner über die Follow-Me-Funktion, wobei Parameter wie Verfolgungsabstand, Höhenversatz und Aktualisierungsrate konfiguriert werden. 

\coleng

ArduPilot supports advanced behaviors including smooth pursuit, orbiting around the target, and terrain-following when additional sensors are available. Its implementation is suitable for professional and research applications.

\colger

ArduPilot unterstützt fortgeschrittene Verfolgungsverfahren, darunter sanfte Annäherung, Umkreisung des Ziels und Geländefolgeflug bei Verwendung zusätzlicher Sensoren. Die Umsetzung eignet sich für professionelle und forschungsorientierte Anwendungen.

\colende

\renewcommand{\deutschertitel}{Follow-Me mit PX4}
\renewcommand{\englischertitel}{Follow-Me with PX4}
\makroabschnitt
\label{AbschnittFollowMePX4}

PX4 provides a Follow-Me feature primarily through the software QGroundControl and companion-computer applications. The function relies on the transmission of position data from a smartphone or companion system (e.g., Raspberry Pi Compute Module\,4) using MAVLink. PX4 requires a GPS-equipped flight controller with barometer, and magnetic compass. 

\colger

PX4 stellt eine Follow-Me-Funktion bereit, die hauptsächlich über die Software QGroundControl und Companion-Computer-Anwendungen implementiert wird. Die Funktion basiert auf der Übertragung von Positionsdaten eines Smartphones oder eines Companion-Systems (z.B. Raspberry Pi Compute Module\,4) über MAVLink. Für den Follow-Me-Betrieb benötigt PX4 einen GPS-fähigen Flugcontroller mit Barometer und magnetischen Kompass.

\colende

\include{chapter07}

% \backmatter 

% \include{Z_0_004_TechnicalTerms}
% \include{glossary}



% % \cleardoublepage
% \phantomsection
% 
% \apptocmd{\thebibliography}{\csname phantomsection\endcsname
% \renewcommand{\bibname}{References}\addcontentsline{toc}{chapter}{References}\addcontentsline{deutschestoc}{chapter}{Literatur}}{}{}
% 
% \include{bibliography}
% 
% 
% \phantomsection
% \cleardoublepage
% 
% \addcontentsline{toc}{chapter}{Index}
% \addcontentsline{deutschestoc}{chapter}{Index}

% \printindex
% \addcontentsline{deutschestoc}{chapter}{Stichwortverzeichnis}
% \printindex[de]
\end{document}
