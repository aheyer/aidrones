\renewcommand{\deutschertitel}{Stereokameras mit einem Raspberry Pi 5}
\renewcommand{\englischertitel}{Stereo camera setup with a Raspberry Pi 5}
\chapter[\texorpdfstring{\protect{\vspace{2pt}\englischertitel}}{\englischertitel}]{}
\kapitel{\deutschertitel}
\thispagestyle{empty}

\label{KapitelStereoVision}

\begin{paracol}{2}[]

{\raggedright\huge\bfseries\sffamily \englischertitel \par\ } \\[1.8ex]

\switchcolumn

{\raggedright\huge\bfseries\sffamily \deutschertitel \par\ } \\[1.8ex]

\coleng

Test Eng

\colger

1. Warum Stereo Vision und was kann man damit machen.  
2. Warum Raspberry Pi 5 (2 RPi-Cams (platzsparend) die gleichzeitig Bilder machen.)
3. Was für ein RPi 5 für die Tests, welche RPi-Cam wurde verwendet (Bild der Konstruktion)
4. Ist nur für große Drohnen geeignet, wegen des hohen Stromverbrauchs.
5. Alternativ gibt es auch dedizierte Stereokameramodule.

\colende

\renewcommand{\deutschertitel}{Camera Calibration}
\renewcommand{\englischertitel}{Kamerakalibrierung}
\makroabschnitt
\label{AbschnittStereoVisionKalibrierung}

Test Eng

\colger

1. Wozu dient die Kamerakalibrierung?
2. Wie läuft die Kalibrierung eines Stereokamerasetups ab (erst Single-Cam dann Stereo)?
3. Was ist dabei zu beachten (Kalibrierungsmuster, Distanz, Auflösung)
4. Wie erstelle ich ein Kalibrierungsmuster und warum ChAruCo?
5. Kalibrierung der individuellen Kameras (+ test mit individuellem entzerren.)
6. Stereo Kalibrierung
7. Undistort Rectify Map 
8. Speichern der Kalibrierungseinstellungen

\colende

\renewcommand{\deutschertitel}{Depth Estimation}
\renewcommand{\englischertitel}{Tiefenerkennung}
\makroabschnitt
\label{AbschnittStereoVisionTiefenerkennung}

Test Eng

\colger

1. Creating a disparity map and tuning it with Block Matching and Semi Global Block Matching
2. From disparity map to depth map.

\colende