\setcounter{page}{5}


\renewcommand{\deutschertitel}{Vorwort}
\renewcommand{\englischertitel}{Preface}

\chapter*{}

% \end{paracol}
\begin{paracol}{2}[]

{\raggedright\huge\bfseries\sffamily \englischertitel \par\ } \\[1.8ex]

\switchcolumn

{\raggedright\huge\bfseries\sffamily \deutschertitel \par\ } \\[1.8ex]

\switchcolumn*
\selectlanguage{english}

TBD

\switchcolumn
\selectlanguage{ngerman}

Dieses Dokument bietet einen Einstieg in das komplexe Thema Drohnen mit künstlicher Intelligenz. Schwerpunkte sind die Entwicklung (inkl. Auswahl geeigneter Hard- und Softwarekomponenten), Bau und Betrieb von Drohnen in der Lehre und für Forschungsprojekte.


\switchcolumn*
\selectlanguage{english}

TBD

\switchcolumn
\selectlanguage{ngerman}


Beim Schreiben dieses Dokuments flossen Erkenntnisse aus dem vom Connectom Vernetzungs- und Innovationsfond des hessian.AI geförderten Forschungsprojekt \textsl{KI-gestützte Drohnenplattform} und aus der Lehrveranstaltung \textsl{Drohnen mit Künstlicher Intelligenz} an der Frankfurt University of Applied Sciences an.


\switchcolumn*
\selectlanguage{english}

TBD

\switchcolumn
\selectlanguage{ngerman}


Maßgebliche Kriterien der Auswahl der in dieses Dokument vorgestellten Komponenten sind unter anderem:

\switchcolumn*
\selectlanguage{english}

TBD

\switchcolumn
\selectlanguage{ngerman}

\begin{itemize}
\item Anschaffungspreis
\item Anpassbarkeit für verschiedenste Einsatzszenarien 
\item Robustheit
\item Langfristige Marktverfügbarkeit
\item Qualität der Dokumentation und Herstellersupport 
\end{itemize}


\end{paracol}

{\vspace{\baselineskip}}%

\textit{Prof.~Dr.~Christian Baun}


